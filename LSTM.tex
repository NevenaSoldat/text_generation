% !TEX encoding = UTF-8 Unicode
\documentclass[a4paper]{article}

\usepackage{color}
\usepackage{url}
\usepackage[T2A]{fontenc} % enable Cyrillic fonts
\usepackage[utf8]{inputenc} % make weird characters work
\usepackage{graphicx}

\usepackage[english,serbian]{babel}
%\usepackage[english,serbianc]{babel} %ukljuciti babel sa ovim opcijama, umesto gornjim, ukoliko se koristi cirilica

\usepackage[unicode]{hyperref}
\hypersetup{colorlinks,citecolor=green,filecolor=green,linkcolor=blue,urlcolor=blue}

\usepackage{listings}

%\newtheorem{primer}{Пример}[section] %ćirilični primer
\newtheorem{primer}{Primer}[section]

\definecolor{mygreen}{rgb}{0,0.6,0}
\definecolor{mygray}{rgb}{0.5,0.5,0.5}
\definecolor{mymauve}{rgb}{0.58,0,0.82}

\lstset{ 
  backgroundcolor=\color{white},   % choose the background color; you must add \usepackage{color} or \usepackage{xcolor}; should come as last argument
  basicstyle=\scriptsize\ttfamily,        % the size of the fonts that are used for the code
  breakatwhitespace=false,         % sets if automatic breaks should only happen at whitespace
  breaklines=true,                 % sets automatic line breaking
  captionpos=b,                    % sets the caption-position to bottom
  commentstyle=\color{mygreen},    % comment style
  deletekeywords={...},            % if you want to delete keywords from the given language
  escapeinside={\%*}{*)},          % if you want to add LaTeX within your code
  extendedchars=true,              % lets you use non-ASCII characters; for 8-bits encodings only, does not work with UTF-8
  firstnumber=1000,                % start line enumeration with line 1000
  frame=single,	                   % adds a frame around the code
  keepspaces=true,                 % keeps spaces in text, useful for keeping indentation of code (possibly needs columns=flexible)
  keywordstyle=\color{blue},       % keyword style
  language=Python,                 % the language of the code
  morekeywords={*,...},            % if you want to add more keywords to the set
  numbers=left,                    % where to put the line-numbers; possible values are (none, left, right)
  numbersep=5pt,                   % how far the line-numbers are from the code
  numberstyle=\tiny\color{mygray}, % the style that is used for the line-numbers
  rulecolor=\color{black},         % if not set, the frame-color may be changed on line-breaks within not-black text (e.g. comments (green here))
  showspaces=false,                % show spaces everywhere adding particular underscores; it overrides 'showstringspaces'
  showstringspaces=false,          % underline spaces within strings only
  showtabs=false,                  % show tabs within strings adding particular underscores
  stepnumber=2,                    % the step between two line-numbers. If it's 1, each line will be numbered
  stringstyle=\color{mymauve},     % string literal style
  tabsize=2,	                   % sets default tabsize to 2 spaces
  title=\lstname                   % show the filename of files included with \lstinputlisting; also try caption instead of title
}

\begin{document}

\title{LSTM neuronske mreže nad sekvencijalnim podacima\\ \small{Seminarski rad u okviru kursa\\Računarska inteligencija\\ Matematički fakultet}}

\author{Milena Kurtić, Nevena Soldat\\ mimikurtic67@gmail.com, nevenasoldat@gmail.com}

%\date{14.~april 2015.}

\maketitle

\abstract{
Poseban vid rekurentnih neuronskih mreža, LSTM neuronske mreže (Long Short Term Memory), opisane su u ovom radu. U prvom delu objasnjena je teorijska strana ovog modela, dok je u sledećem objasnjena primena istih nad problemom prepoznavanja teksta. Na kraju, navedene su prednosti i mane, kao i primena i zaključak izveden iz datog primera.

\tableofcontents

\newpage

\section{Uvod}
\label{sec:uvod}

Neuronske mreže (eng. neural networks) predstavljaju najpopularniju i jednu od najprimenjenijih metoda mašinskog učenja. Njihove primene su mnogobrojne i pomeraju domete veštačke inteligencije, računarstva i primenjene matematike. Postoji više vrsta neuronskih mreža: potpuno povezane, konvolutivne, rekurentne, rekurzivne, grafovske neuronske mreže. U vrstu rekurzivnih spada i LSTM neuronska mreza.

Osnovna ideja veštačke neuronske mreže je simulacija velike količine gusto napakovanih, međusobno  povezanih nervih  ćelija u okviru računara, tako da je omogućeno učenje pojmova, prepoznavanje šablona i donošenje odluka na način koji je sličan čovekovom. Suštinski, veštačke neuronske mreže su softverske simulacije, napravljene programirajući obične računare koji rade u uobičajenom režimu sa svojim tranzistorima i serijski povezanim logičkim  kolima, tako da se ponašaju kao da su napravljene od milijardu međusobno povezanih ćelija mozga koje rade paralelno. 

Jedna od primena neuronskih mreža jeste problem predvidjanja sekvenci, odnosno, predvidjanje sledeće vrednosti za zadati ulazni niz. Sekvence nameću važnost redosleda zapažanja podataka kako prilikom treniranja modela tako i prilikom predviđanja.  
 
 
"Learning of sequential data continues to be a fundamental task and a challenge in
pattern recognition and machine learning. Applications involving sequential data
may require prediction of new events, generation of new sequences, or decision
making such as classification of sequences or sub-sequences."

     — On Prediction Using Variable Order Markov Models, 2004.

\section{LSTM }
\label{sec:zakljucak}


\section{Zaključak}
\label{sec:zakljucak}




\addcontentsline{toc}{section}{Literatura}
\appendix
\bibliography{seminarski} 
\bibliographystyle{plain}

\appendix
\section{Dodatak}

\end{document}
